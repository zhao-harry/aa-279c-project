\section{\Large INTRODUCTION}
Our mission will utilize a satellite with synthetic aperture radar (SAR), designed to gather key remote sensing and environmental data for the Earth. The satellite will be in low Earth orbit (LEO) and use quaternions to describe its orientation, avoiding gimbal lock effects of other conventions. For state estimation, the spacecraft will require gyroscopes, star trackers, and a potentially a sun sensor. For actuation, the spacecraft will likely utilize thrusters, reaction wheels, and magnetorquers.

\subsection{Literature Review}
Space agencies such as NASA have been constructing SAR satellites to gather satellite images and data of Earth for over a decade. Additionally, there exist commercial entities also utilizing SAR in their spacecraft.

For example, Soil Moisture Active Passive (SMAP) is a NASA satellite launched in 2015 that utilizes L-band synthetic aperture radar (SAR) technology to measure soil moisture from LEO. This data has applications in climate change research climate change research applications (such as updating climate models) and some day-to-day activities (such as improving weather forecasts). SMAP is unique in that it had a large deployable reflector, held above the spacecraft body by a deployable boom \cite{SMAP}.

Companies EOS and Capella Space are also developing satellites that use SAR technology in the X-band and S-band frequencies for commercial applications ranging from agriculture to mining \cite{EOSSAR, Capella}. The commercial applicability of SAR is substantial, especially as SAR can penetrate cloud cover while generating high-resolution data, making it superior to many other forms of remote sensing technology. EOS claims to obtain resolution of up to 0.25 m, while Capella Space claims a capability of up to 0.5 m. These satellites all operate in LEO, which enables high-frequency monitoring of the Earth's surface.

NASA and ISRO have partnered to create a SAR satellite as well. The joint project between NASA JPL and ISRO has resulted in the NASA-ISRO Synthetic Aperture Radar (NISAR), a satellite that captures data in the L-band and S-band SAR frequencies \cite{NISARMission}. NISAR's high resolution will permit the detailed measurement of the Earth's surface, enabling better observation of changes in Earth's crust for disaster prevention and mitigation. NISAR will also support science goals such as monitoring ice sheets and the oceans, and its orbit is designed to cover the entire Earth every 12 days.

\subsection{Mission Selection}
NISAR is a joint Earth-observation satellite mission between NASA and ISRO. It is the first satellite to operate in two different Synthetic Aperture Radar (SAR) bands, incorportating both L- and S-band SAR instruments. Both frequencies can penetrate clouds for reliable data collection, but the L-band can also penetrate thicker vegetation that the S-band cannot. Uniquely, NISAR is intended to be used for a wide range of science objectives, including disaster response and agriculture \cite{NISARApps}.