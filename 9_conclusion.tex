\section{\Large CONCLUSION}

\subsection{Summary}
The objective of this paper was to develop an ADCS system for a satellite based on NASA and ISRO's joint NISAR mission. NISAR is a satellite with a synthetic aperture radar (SAR) in a sun-synchronous low Earth orbit (LEO) intended to gather data about Earth's surface. Specifically, science operations require the radar to be pointed at Earth precisely ($273$ arcseconds or $0.075 \degree$ of pointing accuracy). To investigate the ADCS system for such a mission, we performed the following tasks:
\begin{itemize}
  \item Identified specifications and physical characteristics
  \item Discretized the satellite to determine mass and surface properties
  \item Modeled the dynamics of the satellite using mass properties
  \item Propagated dynamics and kinematics using models and analyzed stability
  \item Introduced perturbations based on satellite properties and environmental factors
  \item Investigated attitude determination and Kalman filtering techniques
  \item Simulated controller-in-the-loop operation
  \item Modeled reaction wheel desaturation using magnetorquers
\end{itemize}
We used numerical simulation in MATLAB and Simulink extensively in this paper. Importantly, we demonstrated that the nominal operating attitude for SAR operation was an unstable equilibrium, and we designed an attitude determination and control system that was able to stabilize the satellite to within the pointing accuracy constraints in the presence of realistic modeled disturbance torques. We also showed that the magnetorquer system on board can successfully desaturate the reaction wheels, allowing extended on-orbit operation.

Much additional work can be done in examining additional modes of operation such as slew and detumble. This work includes attitude determination and control design, especially for cases of nonlinearity. Electrical power and thruster analyses will also be important for planning real-world operations.

\subsection{Lessons Learned}
Developing a simulated model of NISAR was an incredible learning experience for the authors. We learned a full-stack approach to ADCS design: orbits, dynamics, kinematics, disturbance modeling, state estimation, actuator modeling, and control. It became clear to us from the AA 279C lectures that there are a lot of tools in ADCS design, and using them in the project was a great way to learn through application.

We also learned much about the software tools used in generating the simulations and report. MATLAB and Simulink were used extensively for every task, and adapting our initially MATLAB-based model to Simulink was a great way to build our first major model in Simulink. Git version control has been extremely important for managing large volumes of code and \LaTeX{} files–integration of code, image generation, and \LaTeX{} contributed significantly to the authors' ability to write a paper of this quality.

Perhaps the most important lesson learned was about the kind of decision-making that lies at the heart of engineering. Throughout this project, many key decisions were made about what exactly to model and the fidelity of the models required. In these decisions, we would need to weigh the benefits of increasing model fidelity (often the accuracy of our simulation) versus the costs (often computational resources and engineering time).