\section{\Large CONCLUSION}

\subsection{Review}
The mission's goal was to develop an ADCS system for a satellite based on NASA and ISRO's joint NISAR mission. NISAR is a satellite with a synthetic aperture radar (SAR) in a sun-synchronous low Earth orbit (LEO) intended to gather data about Earth's surface. NISAR's ADCS system requires $273$ arcseconds ($0.075 \deg$) for pointing, and was required to be constantly pointed at the Earth and uses a combination of reaction wheels, magnetorquers, and thrusters to accomplish this. A Simulink model was developed to model the satellite's general orbital dynamics, disturbance torques, attitude over time, attitude estimation system, and control laws. The modeled system met pointing requirements throughout multiple orbits. While this report outlines a great starting point for an ADCS system, much work remains to be done. A more complicated control law that integrates magnetorquers, reaction wheels, and thrusters to prevent wheel saturation while staying within pointing requirements is a potential area of investigation. Additionally, work could be done to investigate slew maneuvers the satellite would need to perform to go from it's initial attitude after launch to its ideal one.

\subsection{Lessons Learned}
Developing this model was an incredible learning experience for us. For one, we learned a lot about what goes into an ADCS system: the Euler equations, orbital dynamics and kinematics, disturbance torques, ADCS sensors and actuators, attitude determination methods, control laws, and much more. Importantly, it became clear while working on this that there are a lot of tools in the engineering toolbox in ADCS system design and some of the thought processes that are key in deciding which tools to use. Additionally, we were able to greatly familiarize ourselves with developing these kinds of simulations using Simulink. While we initially tried to stick with MATLAB scripts at first, it became clear over time how powerful a tool Simulink can be in developing these kinds of simulations. Perhaps the most important lesson learned was about the kind of decision-making that lies at the heart of engineering. Throughout this project, many key decisions were made about what exactly to model and the fidelity of the models required. In these decisions, we would need to weigh the benefits of increasing model fidelity (often the accuracy of our simulation) versus the costs (often computational resources and engineering time).