\section{\Large PROBLEM SET 7}
\subsection{PROBLEM 1}
\textit{Introduce representative (from manufacturer or Wertz or first lectures) sensor errors in the form of constant bias and Gaussian noise with given standard deviation.}

\subsection{PROBLEM 2}
\textit{Re-apply the attitude determination algorithms from the previous pset. Plot attitude estimation error. Note that the attitude estimation error represents a rotation matrix (DCM) which quantifies how far the estimated attitude is from the true attitude. You can use any parameterization to plot the attitude estimation errors corresponding to this DCM. Is the result consistent with the sensor bias and noise you have introduced?}

\subsection{PROBLEM 3}
\textit{For small sensor errors, the DCM corresponds to a small rotation. Can you give an interpretation of small angles (e.g., in Euler angles and quaternions) to the obtained error DCM?}

\subsection{PROBLEM 4}
\textit{Start modeling actual sensors in dedicated (Simulink or otherwise) subsystems which are part of the spacecraft. These models take inputs from ground-truth simulation and provides output measurements, including systematic and random errors. Take inspiration from overview of sensors discussed in class and textbook for typical errors.}

\subsection{PROBLEM 5}
\textit{Designing and implement the time update of a KF/EKF to obtain the best estimate of the state from the available measurements and models:}

\textit{Search in literature, define, and code a state transition matrix $\Phi$ which provides your state at step k+1 based on the state at step k. Verify that the output of this propagation step is consistent with the rigorous propagation of the attitude (numerical integration). Plot propagation errors as needed.}

\textit{Search in literature, define, and code a control input matrix B which provides the increment to your state at step k+1 due to a control torque at step k. Hint: optional at this stage since you do not have a controller yet.}

\textit{(a) and (b) allow you to propagate the state from k to k+1 including the known control input torques. Hint: Initially you will design the filter by neglecting any control torque from your simulation.}

\textit{Define and code an initial state error covariance matrix P which quantifies the uncertainty of your initial state. This can be picked as diagonal matrix with diagonal elements representing the variance of each state parameter $\sigma^{2}$. Initially you can neglect cross-covariance terms assuming that errors of various state components are not correlated.}

\textit{The time update of the EKF needs $\Phi$, B, and P. Hint: You could increment your navigation performance by keeping the filter receptive to new measurements at steady state through the addition of constant process noise Q at each step. Initially you can define Q similar to P but much smaller (e.g., 1/10 or 1/100).}

\subsection{PROBLEM 6}
\textit{Produce plots showing true attitude estimation errors (estimate vs truth with statistics), formal or estimated attitude estimation errors (covariance from filter). Discuss the results, do they meet expectations? How well is the true estimation error described by the formal covariance? Note that we are only implementing the time update even if we call them “attitude estimates and estimation errors”.}