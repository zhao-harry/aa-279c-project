\section{\Large PROBLEM SET 8}
\subsection{PROBLEM 1}
\textit{The time update should be complete and verified at this stage. We need to include the incoming measurements. Search in literature or derive, define, and code a sensitivity matrix \textbf{H} which provides your state at step k+1 from your measurement vector at step k+1 (i.e., at the same current time, propagation has already been done).}

The following equations define the sensitivity matrix $\mathbf{H_k}$, for the case where there are $n$ vector measurements (sun sensors, star trackers, etc.), and 1 gyroscope vector reading.

\begin{align*}
    \mathbf{H} = \frac{\delta \mathbf{h}}{\delta \mathbf{x}} =
    \begin{bmatrix}
        [\mathbf{h_1} \times] & \mathbf{0}_{3 \times 3} \\
        [\mathbf{h_2} \times] & \mathbf{0}_{3 \times 3} \\
        ... & ... \\
        [\mathbf{h_n} \times] & \mathbf{0}_{3 \times 3} \\
        \mathbf{0}_{3 \times 3} & \mathbf{I}_{3 \times 3} \\
    \end{bmatrix}
\end{align*}

In the equation above, $[\mathbf{h_i} \times]$ refers to the cross-product matrix for the measurement vector $\mathbf{h_i}$.

\subsection{PROBLEM 2}
\textit{Define and code your constant measurement error covariance matrix \textbf{R} which quantifies the uncertainty of your measurements. This can be picked as diagonal matrix with diagonal elements representing the variance of each parameter $\sigma^{2}_{m}$. Hint: If your instrument is well characterized you can use $\sigma_{m}$ applied in your simulation to generate the measurements from your ground-truth.}

For the sensors used in our measurements, $\mathbf{R}$ was defined as a diagonal matrix of the sensor variances discussed in Problem Set 7 Problem 1 (where the sensor errors are the standard deviations of the white noise in each case).

\begin{align*}
    \mathbf{R} = 
    \begin{bmatrix}
        diag(\sigma_{star tracker}^2) & \mathbf{0} & \mathbf{0} \\
        \mathbf{0} & diag(\sigma_{sun sensor}^2) & \mathbf{0} \\
        \mathbf{0} & \mathbf{0} & diag(\sigma_{gyroscopes}^2)\\
    \end{bmatrix}
\end{align*}

\subsection{PROBLEM 3}
\textit{Compute your modelled measurement vector at step k+1 from your state at step k+1. This transformation can be rigorous (non-linear, EKF) or approximate (linear, KF).}

The modeled measurement vector at step k+1 was found by plugging the estimated state at step k+1 into the previously developed measurement model, without measurement noise.

\subsection{PROBLEM 4}
\textit{Compute your pre-fit residuals \textit{z} by differencing modelled and actual measurement vector at k+1.}

The pre-fit residuals were calculated based on the output of the measurements acquired with noise and the measurements from the measurement models.

\subsection{PROBLEM 5}
\textit{The measurement update of the EKF needs \textbf{H}, \textbf{P}, \textbf{R}, and \textbf{z} to compute the Kalman gain \textbf{K}, the new estimated state and its associated covariance matrix.}

The following equations are used in the measurement step calculation.

\begin{align*}
    \mathbf{x_{k+1 | k+1}} = \mathbf{x_{k+1 | k}} + \mathbf{K_k}
    (\mathbf{y_k} - \mathbf{z_k}) \\
    \mathbf{P_{k+1 | k+1}} = \mathbf{P_{k+1 | k}} - 
    \mathbf{K_k} \mathbf{H_k} \mathbf{P_{k+1 | k}} \\
    \mathbf{K_k} = \mathbf{P_{k+1 | k}} \mathbf{H_k}^T [\mathbf{H_k} \mathbf{P_{k+1 | k}} \mathbf{H_k}^T + \mathbf{R_k}]^{-1}
\end{align*}

We implement the measurement update in code with the following function. This MATLAB function is used as part of our Simulink model.

\lstinputlisting{src/measurementUpdate.m}


\subsection{PROBLEM 6}
\textit{Compute your post-fit residuals z by differencing modelled and actual measurement vector at k+1 using your new state. These should be smaller than the pre-fit residuals and should capture the standard deviation of your measurements at steady state.}

The post-fit residuals z are found by taking the updated state from the measurement update and plugging them into the noiseless measurement model again. Figures \ref{fig:ps8_problem7_res_units} and \ref{fig:ps8_problem7_res_gyro} show the plots of the post-fit and pre-fit residuals.

\subsection{PROBLEM 7}
\textit{Produce plots showing true attitude estimation errors (estimate vs truth with statistics at steady state), formal or estimated attitude estimation errors (covariance from filter), pre- and post-fit residuals (with statistics at steady state), etc. Discuss the results, do they meet expectations? Is the true estimation error well described by the formal covariance? Are the measurements residuals consistent with the applied measurement errors? Hint: show estimation errors, do not overlap state estimate with reference truth.}

Below are the error and residual plots. Figure \ref{fig:ps8_problem7_error} shows the attitude estimate errors with the MEKF method. As expected, the errors are incredibly small and beat out the other attitude determination methods used in the previous section.

Figure \ref{fig:ps8_problem7_cov} shows the estimated attitude determination errors based on the covariance of the filter. Comparing this with Figure \ref{fig:ps8_problem7_error}, it seems that this method tends to under-report the error and is overall not incredibly accurate.

Figures \ref{fig:ps8_problem7_res_units} and \ref{fig:ps8_problem7_res_gyro} below show the difference in the norms of the pre-fit and post-fit residuals. As expected, the post-fit residuals are slightly smaller than the pre-fit residuals. However, the difference between pre-fit and post-fit sensor noise is much better in the gyroscope case than the unit vector case. This is something that could be improved in the future.

\begin{figure}[H]
\centering
\includegraphics[scale=0.6]{Images/ps8_problem7_error.png}
\caption{Errors between MEKF measurements and ground truth}
\label{fig:ps8_problem7_error}
\end{figure}

\begin{figure}[H]
\centering
\includegraphics[scale=0.6]{Images/ps8_problem7_cov.png}
\caption{Estimated attitude estimation errors from covariance}
\label{fig:ps8_problem7_cov}
\end{figure}

\begin{figure}[H]
\centering
\includegraphics[scale=0.8]{Images/ps8_problem7_res_units.png}
\caption{Norm of pre-fit and post-fit vector-based residuals}
\label{fig:ps8_problem7_res_units}
\end{figure}

\begin{figure}[H]
\centering
\includegraphics[scale=0.6]{Images/ps8_problem7_res_gyro.png}
\caption{Pre-fit and post-fit gyroscope residuals}
\label{fig:ps8_problem7_res_gyro}
\end{figure}

\subsection{PROBLEM 8}
\textit{Start thinking/planning possible upgrades for the final project deliverable. Upgrades can go in several directions tailored to your project needs. For example, define different modes for attitude estimation using different sensors and algorithms based on your concept of operations. What would it take to implement a UKF instead of an EKF? Can you improve your dynamics and measurement models? Can you use measurements that are more representative of what your sensors are going to actually provide you? Hint: you should not panic if your Kalman filter is not working, you can always address the rest of the problem sets bypassing the Kalman filter. Do not give up though!}

One possible upgrade for the final project could be to include biases in the EKF state such that we can track drift. Strictly adhering to the assumptions of the Kalman filter, we should have unbiased measurements in our system, and we should be correcting the value of the measurements which is passed into the filter using the bias or drift which we track. However, this would require modeling the dynamics of the drift in order to simulate properly.

We also have an opportunity to model the magnetic field–we have already implemented a fourth-order magnetic field model, but additional refinements and fixes could improve this model and make it usable for attitude determination in conjunction with a magnetometer. We can also take advantage of the magnetic field for actuation via magnetorquers.

An interesting topic to study is whether our system is able to accurately predict the attitude of the satellite within the pointing requirements necessary for NISAR to successfully collect science data. These requirements are available in literature for comparison, and we can also examine the effect of this on the image swath captured by the SAR sensor suite.