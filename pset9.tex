\section{\Large PROBLEM SET 9}
\subsection{PROBLEM 1}
\textit{Start selecting and sizing your actuators using the basic dimensional formulas provided in class (Wertz,SMAD) based on your perturbation environment and attitude slew objectives. Provide rationale for choice.}

We size reaction wheels, magnetorquers, and thrusters in this section. For this problem set, we only simulate the reaction wheels, but we may include magnetorquers and thrusters later.

For reaction wheels, we referenced the ASTROFEIN RW3000, which is very similar to the nominal angular momentum of the reaction wheels used by NISAR. These reaction wheels produce an angular momentum of 50 Nms and 4000 rpm.

\begin{table}[H]
\begin{tabular}{|l|l|l|l|}
\hline
Reference & Momentum & Moment of Inertia             & Maximum rpm \\ \hline
RW 3000   & 50 Nms   & 0.119 kg*m\textasciicircum{}2 & 4000        \\ \hline
\end{tabular}
\caption{}
\end{table}



\subsection{PROBLEM 2}
\textit{Start modeling actuators in dedicated (Simulink) subsystems which are part of the spacecraft. These models take inputs from ground-truth simulation and provides output control torques to be directly fed into the Euler equations, including systematic and random errors. Plot output vs input of actuator model. Hint: for the rest of this problem set you should remove all sensing and actuation errors for verification purposes.}

Our actuator model accepts $M_{c}$ as a torque command from the control law and tracks the angular momentum state of our reaction wheels. We can propagate the angular momentum by computing the rate of change of the angular momentum. This is also equivalent to commanding a change in angular velocity of the reaction wheels.
\begin{align*}
    \Dot{\Vec{L}}_{w} &= \Vec{A}^{*} (-\Vec{M}_{c} - \Vec{\omega} \times \Vec{A} \Vec{L}_{w})
\end{align*}

We can incorporate estimation error by passing in $\omega_{est}$ when computing $\Dot{\Vec{L}}_{w}$, and then propagating the dynamics using $\omega_{true}$ in the same formula for $\Vec{M}_{c}$.

Prior to implementing the control law, we use an arbitrary sinusoidal signal as our torque command and plot the resulting output torque.

\subsection{PROBLEM 3}
\textit{Start designing and implementing a linear control law which provides the desired behavior of the dynamic system (damping and frequency) by a feedback done on the control tracking errors. Try using both a small angle approximation and a non-linear approach in defining the control tracking errors used in the control law. Hint: this task requires that you linearize the Euler equations and that you show how the gains of the linear control law are derived, also start applying the control law by assuming that the actuators are doing their job ideally (no actuation equations, no saturation or other limits). Selection of gains can be done through standard PD, pole placement or LQR approaches.}

We implement the control law using the following equations, where $f$ is frequency of the actuator response and $I_{i}$ is the moment of inertia about principal axis $i$. The control law differs slightly for each axis based on the moment of inertia about that axis.

\begin{align*}
    M_{c,i} &= -K_{p,i} \alpha_{i} - K_{d,i} \Dot{\alpha}_{i} \\
    K_{p,i} &= \frac{f^{2}}{I_{i}} \\
    K_{d,i} &= 2 \sqrt{I_{i} \left[ 3 n^{2} (I_{k} - I_{i}) + K_{p,i} \right]}
\end{align*}

\subsection{PROBLEM 4}
\textit{Start plotting all the relevant quantities of your simulation, including attitude determination errors, attitude control errors, control actions (along all axes), etc. Hint: stress difference between what the ADCS believes, and what the spacecraft is actually doing. This affects estimation, control errors, and control actions.}

